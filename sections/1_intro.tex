\begin{multicols}{2}
\tableofcontents
\section{Introduction}
Cloud computing evolved to support scalable platform usage, moving from single-file applications to client/server 
architectures with monolithic backends, then to microservices with containers, and now transitioning to micro-frontends. 
Containers run directly on the host using Linux CGroups, bypassing the hypervisor overhead inherent in VMs (e.g., 
Proxmox, VMWare), which allows for full utilization of host CPU resources and simplifies telemetry monitoring. 
This lightweight nature is evident in Docker files that simply copy files and run builds, making containers easier to integrate 
into CI/CD pipelines.

However, this increased reliance on cloud-based infrastructures has led to a significant rise in energy 
consumption. In 2006, the electricity costs for IT infrastructures in the United States alone were 
estimated at \$4.5 billion, with projections to double by 2011\cite{beloglazov_energy_2010}. Energy 
consumption optimization has since become a critical concern, especially as cloud data centers now account 
for approximately 1-1.5\% of global electricity use\cite{IEADataCentres}. Despite efficiency improvements, the 
demand for digital services continues to grow, pushing the need for more sustainable solutions.

Early research focused on heuristic-based approaches to optimize virtual machine (VM) placement, achieving 
energy savings of up to 83\% while maintaining only a 1.1\% service level agreement (SLA) violation 
rate\cite{beloglazov_energy_2010}. More recently, research has shifted from VM-based allocation towards 
containerized environments, where energy efficiency is influenced by scheduling strategies, workload 
distribution, and infrastructure optimizations. Studies indicate that modern cloud providers, including 
Amazon, Microsoft, Google, and Meta, have doubled their energy consumption between 2017 and 2021, 
reaching 72 TWh in 2021\cite{masanet_2020, hintemann_2022, IEADataCentres}.

% Recent research has begun to explore renewable energy-aware strategies to further optimize cloud infrastructures. A notable study proposed a renewable energy-based multi-indexed job classification and scheduling scheme using containers for sustainability in cloud data centers\cite{kumar_renewable_2019}. The scheme focuses on transferring workloads to data centers with sufficient renewable energy, incorporating energy-efficient server selection and container consolidation. The results showed that the proposed method achieved 15\%, 28\%, and 10.55\% higher energy savings compared to existing approaches, demonstrating the potential for significantly reducing energy consumption while enhancing sustainability.

The paper ``Survey on Energy Consumption Optimization Approach in Container Based Cloud Environments'' 
further highlights that containerization not only drives scalability and 
reproducibility but also plays a crucial role in optimizing energy consumption. It 
explores strategies for efficient resource allocation, reducing power overhead, and 
ensuring that the benefits of container-based deployments extend beyond performance 
to sustainability in cloud infrastructures.

In this work, we present a state-of-the-art review on Energy Consumption Optimization Approaches in Container-Based Cloud Environments. Our survey of the available literature—predominantly spanning from 2010 to 2020—reveals that foundational research primarily focused on energy measurement, basic optimization strategies, and energy visualization techniques\cite{beloglazov_energy_2010}.

Early contributions, such as those by Beloglazov and Buyya\cite{beloglazov_energy_2010}, as well as 
Piraghaj et al.\cite{piraghaj_framework_2015}, laid the groundwork for dynamic resource allocation and 
energy-efficient container consolidation. Later advancements introduced more sophisticated container 
scheduling mechanisms, including availability-aware scheduling\cite{alahmad_availability-aware_2018}, 
concurrent scheduling in heterogeneous clusters\cite{hu_concurrent_2020}, and hybrid AI-driven resource 
allocation\cite{tan_hybrid_2019}.

In parallel, energy-efficient resource management techniques gained prominence, incorporating renewable 
energy-aware scheduling\cite{kumar_renewable_2019}, optimization-based consolidation methods\cite{shi_energy-aware_2018, piraghaj_framework_2015}, 
and brownout-based scheduling strategies\cite{xu_energy_2016}. Several studies further explored 
predictive optimization and SLA-aware provisioning frameworks to enhance energy 
efficiency\cite{dabbagh_energy-efficient_2015, hameed_survey_2016, li_sla-aware_2018, bui_energy_2017, 
carrega_energy-aware_2017}.

Beyond optimization techniques, researchers have also examined broader energy consumption trends and policy implications\cite{avgerinou_trends_2017}, reflecting the increasing emphasis on sustainability in cloud computing. Additionally, efforts in DevOps-driven elastic container management have contributed to improving the adaptability and efficiency of containerized cloud applications\cite{barna_delivering_2017}.

Although these prior works have significantly advanced the field, our review highlights an evolving trend toward more integrated and user-centric energy management strategies, as reflected in recent data on energy consumption in cloud environments\cite{masanet_2020, hintemann_2022, IEADataCentres}.

\end{multicols}
