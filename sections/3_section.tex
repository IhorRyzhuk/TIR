{\color{gray}\hrule}
\begin{center}
\section{Trend reconstruction}
\textbf{Chronological evolution of the trend}
\bigskip
\end{center}
{\color{gray}\hrule}
\begin{multicols}{2}
\subsection{Early approaches}
Early approaches date back to 2010, with Docker being released in 2013. In 2006, the cost of electricity consumed by IT infrastructures in the US was estimated at 4.5 billion dollars and was projected to double by 2011 \cite{beloglazov_energy_2010}.

In the initial attempts described in \cite{beloglazov_energy_2010}, a simple Bin Packing variation with DVFS enabled was used to address this problem. The results, obtained using the CloudSim environment, showed an energy savings gain of 83\% compared to no policy, with an SLA violation of 1.1\%. This marked the beginning of research in this area.

By 2015, researchers shifted toward predictive modeling to overcome the limitations of static thresholds. In \cite{dabbagh_energy-efficient_2015}, a Wiener filter-based predictor was introduced to estimate cluster workloads, combined with Best Fit Decreasing for physical machine (PM) allocation. This approach improved energy efficiency by up to 33\% compared to heuristic methods.
Shortly after, in 2017, further improvements were made to predictive models \cite{bui_energy_2017}. This work extended previous research by incorporating Gaussian Process Regression (GPR) for non-stationary workload prediction, accelerated using the Fast Fourier Transform (FFT) to mitigate GPR's computational complexity. Additionally, a convex optimization-based migration strategy was employed. These enhancements resulted in a 35\% energy efficiency improvement over heuristic methods, with only a 15\% latency tradeoff.

\subsection{Containers Rise}
Containers began replacing VMs for lightweight virtualization. Initial efforts focused on container migration to reduce active VM count. Strategies involved modular watchdogs and Pearson correlation checks. [Notes - Long Version.pdf]

Flow network models enabled scalable concurrent container scheduling with multi-resource awareness and affinity considerations. These models balanced execution speed and resource utilization across up to 5000 machines. [Notes - Long Version.pdf]


Focus shifted to application-layer availability using metrics like MTTF and MTTR. Scheduling considered redundancy, affinity, and anti-affinity rules to ensure SLA compliance during failures. [Notes - Long Version.pdf]

\end{multicols}


